\documentclass[epic,eepic,aspectratio=169,12pt]{beamer}
\usepackage[polish]{babel}
\usepackage[T1]{polski}
\usepackage[utf8]{inputenc}
\usepackage[T1]{fontenc}
\usepackage{color}
\usepackage{picture}
\usepackage{graphicx}
\usepackage{minibox}
\usepackage{csquotes}
\usetheme{Warsaw}
\usecolortheme{crane}

\usepackage[]{csquotes}
\DeclareQuoteAlias{german}{polish}
\usepackage[%style=numeric %,authoryear,  alphabetic, authoryear, ect.
sorting=nty,
isbn=true,
backend=biber]{biblatex}


\begin{document}
	\title{Monitoring aplikacji}
	\subtitle{Prometheus}
	\author{Krzysztof Pobożan}
\begin{frame}
	\maketitle
\end{frame}
\begin{frame}{Agenda}
	\begin{itemize}
		\item Wstęp
		\item Czym jest Prometheus
		\item Rodzaje metryk
		\item Agregacja, Alerty i jak używać metryk
		\item DEMO
	\end{itemize}
\end{frame}
\section{Wstęp}
\begin{frame}{Wstęp}
	Potrzebujemy narzędzia do śledzenia zachowania się systemu.
\end{frame}
\section{Czym jest Prometheus}
\begin{frame}{Czym jest Prometheus}
	\begin{itemize}
		\item Open Source \pause
		\item Monitoring \pause		
		\item Alerty
	\end{itemize}
\end{frame}
\begin{frame}{Możliwośći Prometheusa}
	\begin{itemize}
		\item  wielowymiarowy model danych
		\item  elastyczny język zapytań
		\item nie bazuje na rozproszonym dysku - każdy serwer jest niezależny
		\item serie czasowe zbierane w modelu PULL
		\item cele są wykrywane poprzez Discovery Service lub statycznie.
	\end{itemize}
\end{frame}
\end{document}