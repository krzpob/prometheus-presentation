\documentclass[12pt,a4paper,notitlepage,onecolumn]{article}
\usepackage[T1]{polski}
\usepackage[utf8]{inputenc}
\usepackage[T1]{fontenc}
\usepackage{graphicx}
\author{Krzysztof Pobożan, Marcin Przybylski}
\title{Demo monitoringu na przykładzie środowiska testowego squid i pay}
\begin{document}
	\maketitle
	
	\section{Konsola prometheus}
	(mozna jako pierwsze jesli po architekturze)
	\subsection{źródła metryk}
	http://coreos01.gazeta.pl:9090/targets
	\subsection{agregacje, promQL}
	http://coreos01.gazeta.pl:9090/graph
	Przykłady zapytań jeden z dwa np. czasow zapytan, bledow etc	
	
	\section{Wystawione metryki}
	pigeon
	
	turtle?
	
	ecard-proxy customowe liczniki bledow http przy ecardzie
	
	invoice-proxy, spring boot 1.4 - inna konfiguracja, dodatkowy endpoint pod kolejke od zlecen fakturowacza, zabezpieczenie po basic auth
	
	\subsection{wystartowanie apki}
	wystartowanie apki/endpoint /prometheus w przeglądarce
	widoczne na /graph w prometheusie
	
	\subsection{prezentacja w graphanie}
	omowienie wykresow
	
	
	\section{Alerty}
     Pokazanie konfiguracji alertmanagera ze zdefinowanymi alertami.
     
     Reczne przekroczenie liczby bledow w ecardzie  rest clientem na ecard-proxy lub zapchanie kolejki zlecen w invoice-proxy -
      jednoczesnie pokazac zmianę w graphanie
     
     jakis czasowy?
     
     odebranie maila/powiadomienia na slacku
     
\end{document}